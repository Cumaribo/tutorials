\PassOptionsToPackage{unicode=true}{hyperref} % options for packages loaded elsewhere
\PassOptionsToPackage{hyphens}{url}
%
\documentclass[]{article}
\usepackage{lmodern}
\usepackage{amssymb,amsmath}
\usepackage{ifxetex,ifluatex}
\usepackage{fixltx2e} % provides \textsubscript
\ifnum 0\ifxetex 1\fi\ifluatex 1\fi=0 % if pdftex
  \usepackage[T1]{fontenc}
  \usepackage[utf8]{inputenc}
  \usepackage{textcomp} % provides euro and other symbols
\else % if luatex or xelatex
  \usepackage{unicode-math}
  \defaultfontfeatures{Ligatures=TeX,Scale=MatchLowercase}
\fi
% use upquote if available, for straight quotes in verbatim environments
\IfFileExists{upquote.sty}{\usepackage{upquote}}{}
% use microtype if available
\IfFileExists{microtype.sty}{%
\usepackage[]{microtype}
\UseMicrotypeSet[protrusion]{basicmath} % disable protrusion for tt fonts
}{}
\IfFileExists{parskip.sty}{%
\usepackage{parskip}
}{% else
\setlength{\parindent}{0pt}
\setlength{\parskip}{6pt plus 2pt minus 1pt}
}
\usepackage{hyperref}
\hypersetup{
            pdftitle={GLAD ARD Landsat Tools tutorial},
            pdfauthor={Jeronimo Rodriguez-Escobar},
            pdfborder={0 0 0},
            breaklinks=true}
\urlstyle{same}  % don't use monospace font for urls
\usepackage[margin=1in]{geometry}
\usepackage{graphicx,grffile}
\makeatletter
\def\maxwidth{\ifdim\Gin@nat@width>\linewidth\linewidth\else\Gin@nat@width\fi}
\def\maxheight{\ifdim\Gin@nat@height>\textheight\textheight\else\Gin@nat@height\fi}
\makeatother
% Scale images if necessary, so that they will not overflow the page
% margins by default, and it is still possible to overwrite the defaults
% using explicit options in \includegraphics[width, height, ...]{}
\setkeys{Gin}{width=\maxwidth,height=\maxheight,keepaspectratio}
\setlength{\emergencystretch}{3em}  % prevent overfull lines
\providecommand{\tightlist}{%
  \setlength{\itemsep}{0pt}\setlength{\parskip}{0pt}}
\setcounter{secnumdepth}{0}
% Redefines (sub)paragraphs to behave more like sections
\ifx\paragraph\undefined\else
\let\oldparagraph\paragraph
\renewcommand{\paragraph}[1]{\oldparagraph{#1}\mbox{}}
\fi
\ifx\subparagraph\undefined\else
\let\oldsubparagraph\subparagraph
\renewcommand{\subparagraph}[1]{\oldsubparagraph{#1}\mbox{}}
\fi

% set default figure placement to htbp
\makeatletter
\def\fps@figure{htbp}
\makeatother


\title{GLAD ARD Landsat Tools tutorial}
\author{Jeronimo Rodriguez-Escobar}
\date{9/13/2021}

\begin{document}
\maketitle

\hypertarget{goals}{%
\subsection{Goals:}\label{goals}}

Learn the basics of the ARD data, how to obtain tghe data, extract
phenollogy metrics and build gap filled radiometrically consistent
mosaics

\hypertarget{introduction}{%
\subsection{Introduction}\label{introduction}}

The Global Land Analysis and Discovery (GLAD) team at the University of
Maryland has developed and implemented an automated Landsat data
processing system that generates globally consistent analysis ready data
(GLAD Landsat ARD) as inputs for land cover and land use mapping and
change analysis. The GLAD Landsat ARD represents 16-day time-series of
globally consistent, tiled Landsat normalized surface reflectance from
1997 to present, updated annually, and suitable for operational land
cover change applications

\hypertarget{conversion-to-radiometric-quantity}{%
\subsubsection{Conversion To Radiometric
Quantity}\label{conversion-to-radiometric-quantity}}

In ARD, only spectral bands with matching wavelengths between TM, ETM+,
and OLI/TIRS sensors are processed. For thermal infrared data, a
high-gain mode thermal band (band 62) of the ETM+ sensor and
10.6--11.19μm thermal band (band 10) of the TIRS sensor are used.
Landsat Collection 1 data contain radiation measurements for reflective
visible/infrared bands recorded as integer digital numbers (DNs). This
data is converted into top-of-atmosphere (TOA) reflectance, scaled
across all Landsat sensors. Spectral reflectance (value range from zero
to one) is scaled from 1 to 40,000 and recorded as a 16-bit unsigned
integer value.

\hypertarget{observation-quality-assessmebt}{%
\subsubsection{Observation Quality
Assessmebt}\label{observation-quality-assessmebt}}

The GLAD observation quality assessment model developed represents a set
of regionally adapted decision tree ensembles to map the likelihood of a
pixel to represent cloud, cloud shadow, heavy haze, and, for clear-sky
observations, water or snow/ice. The model outputs represent likelihoods
of assigning a pixel to the cloud, shadow, haze, snow/ice, and water
classes. The masks were subsequently aggregated into an integral
observation Quality Flag (QF) that highlights cloud/shadow contaminated
observations, separates topographic shadows from likely cloud shadows,
and specifies the proximity to clouds and cloud shadows.

\hypertarget{reflectance-normalization}{%
\subsubsection{Reflectance
Normalization}\label{reflectance-normalization}}

\hypertarget{glad-tools-v1.1-installation-windows-10}{%
\subsection{GLAD Tools V1.1 Installation (Windows
10)}\label{glad-tools-v1.1-installation-windows-10}}

\hypertarget{perl}{%
\subsubsection{1. Perl}\label{perl}}

\begin{enumerate}
\def\labelenumi{\arabic{enumi}.}
\item
  Obtain \href{www.strawberryperl.com}{Strawberry Perl} and install the
  64 bit version
\item
  QGIS and OSGeo4W QGIS is used as the source for the GDAL libraries and
  tools that are required forthe ARD tools to run. Additionally it is
  recomended to have following plug-ins (optional for the moment):
\end{enumerate}

\begin{itemize}
\tightlist
\item
  Send2GE (Send to Google Earth)
\item
  Quick Map Services (basemaps)
\end{itemize}

\begin{enumerate}
\def\labelenumi{\arabic{enumi}.}
\setcounter{enumi}{2}
\tightlist
\item
  ARD Tools.
\end{enumerate}

\begin{itemize}
\tightlist
\item
  \href{https://glad.umd.edu/Potapov/ARD/GLAD_1.1.zip}{Download ARD
  Tools}
\item
  Create the folder C:\GLAD\_1.1 in your local machine and decompress
  the zip file in that location.
\item
  Right click on \textbf{ADD\_PATH\_for\_GLAD\_v1.1.bat} and select
  ``Run as administrator''
\item
  Reboot the computer
\end{itemize}

\begin{enumerate}
\def\labelenumi{\arabic{enumi}.}
\setcounter{enumi}{3}
\tightlist
\item
  Navigate to your working folder
\end{enumerate}

\begin{itemize}
\tightlist
\item
  In the Windows command line (CMD) navigate to the directry that you
  are going to use as working folder. It can be in the local machine or
  in a mapped network drive. just use you are going to work. In my
  casze, i am workin on the Y drive
\end{itemize}

\href{}{Change Directory}

\begin{enumerate}
\def\labelenumi{\arabic{enumi}.}
\setcounter{enumi}{4}
\item
  OSGeo4W Locate the OSGeo4w folder. In the machines in the Salt Lab it
  is located here: \href{xx}{``C:\OSGeo4w''}
\item
  Environmental Varibles.
\end{enumerate}

\begin{itemize}
\tightlist
\item
  In the GLAD working folder, run the command ``path'' to make sure that
  the variables were set correclty.
\end{itemize}

It needs to look similaer to this

\hypertarget{select-tiles-for-your-study-area.}{%
\subsection{Select Tiles for your study
Area.}\label{select-tiles-for-your-study-area.}}

Use \textbf{select by location} and select the tiles that intersect with
your study area.

In this case, we´re going to use the Philly Area as an example. Copy the
list in a .txt file and save it in your working folder (\textbf{\emph{it
is not the GLAD\_1.1 folder in the C folder drive}})

074W\_39N\\
075W\_39N\\
074W\_40N\\
075W\_40N

Foe each 1x1 tile, data is sored as om 16 days periods between january
1998 (415) and december (943). Tehre are 23 periods per year.
*\textbf{Note} Not all tiles have valid pixels!!

\hypertarget{download-the-ard-data.}{%
\subsection{Download the ARD data.}\label{download-the-ard-data.}}

Set your tile list, the date interval you want to obtain, and the
destination folder where you want to store the data.

\textbf{\emph{Note:}} Each tile/year requires +- 5GB of storage space,
it adds up fast, be careful. I already downloaded all the tiles for
Colombia, it uses more than 7TB of space.

\begin{verbatim}
perl C:/GLAD_1.1/download_V1.1.pl <<your_username>> <<your_password>> tiles_philly.txt 917 943 C:/ARD_tutorial
\end{verbatim}

\hypertarget{download-the-srtm-data-optional}{%
\subsection{Download the SRTM Data
(optional)}\label{download-the-srtm-data-optional}}

ARD tools offers the possibility to obtain SRTM elevation data for the
same study area. It can be quite handy, including sloep and aspect. The
structure is the same, just change the perl script

\begin{verbatim}
perl C:/GLAD_1.1/download_SRTM.pl <<your_username>> <<your_password>> tiles_philly.txt C:/ARD_tutorial
\end{verbatim}

\hypertarget{glad-data-structure}{%
\subsection{GLAD Data Structure}\label{glad-data-structure}}

\#\#Multi temporal Metric Extraction

This might be the most useful part so far, and the part that we´re going
to use more often. GLAD Tools lets calculate two types of metrics,
\textbf{annual phenological metrics} and \textbf{annual change detection
metrics}, for the two most common objectives: annual land cover mapping
and detection of land cover changes between two consecutive years.

There are 4 types of phenollogical metrics, pheno A, pheno B, pheno C
and pheno D, depending on the mapping objectives, the availalble
computing and strorage capacity. We´re going to focus on pheno C and D,
but the obtention is the same for all cases.

pheno C: standard, richest repository, used for global mapping pheno D:
produces annual cluid free Landsat Image composites. It requires the
less space and computes fastest. Today we´re going to work with them.

\hypertarget{phenological-metrics}{%
\subsection{Phenological Metrics}\label{phenological-metrics}}

For annual land cover and vegetation structure mapping models
extrapolation in space and time. Generated primarily using observations
collected during a given calendar year (January 1 -- December 31). Data
from the previous years may be used to fill gaps. Includes two stages:
(1) select clear-sky observations and fillz gaps in the observation time
series; and (2) extract reflectance distribution statistics from the
selected observation time-series

\begin{enumerate}
\def\labelenumi{\arabic{enumi}.}
\tightlist
\item
  Set the metrics extration parameter file (txt). Is sets the metrics
  yoiu want to extract, the list of tiles, the target year, the location
  of the input data, the folder where the data obtained will be stored,
  the number of cores to process and the number of years that will be
  used to fill the gaps
\end{enumerate}

mettype=pheno\_D\\
tilelist=tiles\_philly.txt\\
year=2020\\
input=C:/ARD\_tutorial\\
output=C:/ARD\_tutorial/Pheno\_d\\
threads=8\\
gapfill=2

run the metrics extraction line in the CMD

\begin{verbatim}
perl C:/GLAD_1.1/build_metrics.pl parameter_philly.txt
\end{verbatim}

\begin{enumerate}
\def\labelenumi{\arabic{enumi}.}
\setcounter{enumi}{1}
\tightlist
\item
  Set the mosaicking parameter file
\end{enumerate}

source=C:/ARD\_tutorial/Pheno\_d\\
list=tiles\_philly.txt\\
year=2020\\
outname=philly\_ARD\\
bands=blue\_av2575, green\_av2575, red\_av2575, nir\_av2575,
swir1\_av2575, swir2\_av2575, TEC\_pf, TEC\_prcwater\\
ogr=C:/OSGeo4w/OSGeo4W.bat

\begin{verbatim}
perl C:/GLAD_1.1/mosaic_tiles.pl param_mosaic_2020.txt
\end{verbatim}

\textbf{\emph{Attention}} \emph{Make sure that the metrics and tiles for
the year you´re building the mosaic are available. }It will add the year
to the file name * the order of the bands in the param file will be
maintained * Having the correct route for the ogr is key for this script
to work properly.

\end{document}
